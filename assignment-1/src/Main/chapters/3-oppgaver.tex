\section{Oppgaver (100 \%) - Grunnleggende Linux}\label{sec:3-oppgave}


\begin{subprob}
    Naviger til rotmappen (\verb|cd /|), og bruk \verb|ls| og \verb|cd| for å komme tilbake til egen hjemmemappe uten å bruke \verb|cd -|.
	\begin{solution}
	    dummy text. dummy text. 
	\end{solution}
\end{subprob}

\begin{subprob}
    I Linux er det mange filer som starter med et punktum i navnet. Ulempen med punktum i navnet er at disse ikke blir vist i filutforskeren eller av \verb|ls| med mindre man spesifikt ber om det. Bruk \verb|man| til å finne hvilket flagg som \verb|ls| krever for å vise alle filene i en mappe. 
	\begin{solution}
	    dummy text. dummy text. 
	\end{solution}
\end{subprob}

\begin{subprob}
    Linux kan \textit{pipe} output fra en kommando inn i en annen kommando. Et eksempel på det er \texttt{cat main.c | less}, hvor returverdien av \verb|main.c| blir gitt som input til kommandoen \verb|less|. Lær mer om \verb|less| med \verb|man| og prøv \texttt{cat main.c | less} med \verb|main.c| i \verb|source|-mappen.
    
	\begin{solution}
	    dummy text. dummy text. 
	\end{solution}
\end{subprob}

\begin{subprob}
    Filutforskerprogrammet som kommer med Ubuntu heter \textit{nautilus}, og kan startes fra terminalen ved å kalle \verb|nautilus|. Prøv å kalle \verb|nautilus . &| hvor punktum blir brukt som argument, mens \verb|&| blir brukt for at terminalen skal starte \textit{nautilus} i bakgrunnen. Bruk \verb|man| til å finne hvilke flagg som kreves for å avslutte \verb|nautilus|.
    \begin{solution}
        dummy text. dummy text. 
	\end{solution}
\end{subprob}

\begin{subprob}
    Opprett en ny fil med \verb|touch| og \textit{pipe} innholdet fra \verb|main.c| inn i den nye filen. Bekreft at du har kopiert innholdet fra \verb|main.c| over til den nye filen ved å bruke tekstredigeringsverktøyet \verb|nano|.
\end{subprob}

\begin{subprob}
    Erstatt en av tekststrengene i \verb|printf| med \textit{Linux er lett!} i filen du opprettet i oppgave \textbf{e}. \textit{Hint:} Bruk \verb|nano|.
\end{subprob}

\begin{subprob}
    Oppgrader alle pakkene på datamaskinen på sanntidssalen med \verb|apt|.
	\begin{solution}
	    dummy text. 
	\end{solution}
\end{subprob}



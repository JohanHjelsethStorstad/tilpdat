
\section*{Revisjonshistorie}
\begin{center}
 \begin{tabular}{|p{1.5cm} p{5.5cm}|} 
 \hline
 År & Forfatter \\ [0.5ex] 
 \hline\hline
 2020 & Kolbjørn Austreng  \\ 
 \hline
 2021 & Kiet Tuan Hoang \\
 \hline
 2022 & Kiet Tuan Hoang \\
 \hline
  2023 & Kiet Tuan Hoang \\
 \hline
 2024 & Terje Haugland Jacobsson\\
 & Tord Natlandsmyr \\
 \hline
\end{tabular}
\end{center}

% \begin{alphasection}
\section{Introduksjon - Praktisk rundt filene}

I denne øvingen får dere utlevert en \verb|.c|-fil i \verb|source|-mappen. Tabellen under lister opp filen som kommer med i \verb|source|-mappen samt litt informasjon om dere skal endre på filen eller om dere skal la den bli i løpet av øvingen.

\begin{center}
    \begin{tabular}{|p{8.5cm} p{5.5cm}|} 
    \hline
    \textbf{Filer} & \textbf{Skal filen(e) endres?}  \\ [0.5ex] 
    \hline\hline
    \verb|source/main.c| & Ja  \\ 
    \hline
    \verb|Main/*| & Nei  \\ 
    \hline
    \verb|.github/*| & Nei \\
    \hline 
    \end{tabular}
\end{center}

\section{Introduksjon - Praktisk rundt øvingen}
Denne øvingen gir en innføring i Linux, slik at resten av labopplegget fremover ikke foregår i et helt ukjent miljø. Seksjon \ref{sec:2-innføring} gir en innføring i grunnleggende kommandoer som er nyttige i Linux, og som trengs for å fullføre oppgavene som følger i seksjon \ref{sec:3-oppgave}. Dere får godkjent øving 1 ved at dere demonstrerer foran en studass at dere har gjort alle oppgavene i seksjon \ref{sec:3-oppgave}. 
I tillegg har to appendikser blitt lagt til (se appendiks \ref{4-appendix} og \ref{app:cygwin}) i tilfelle det er noen som enten har lyst til å installere Linux på egen maskin med dual booting, eller jobbe på et Unix-liknende omgivelse for Windows med Cygwin.



\section{Introduksjon - Litt om GNU/Linux}

Linux (Linux kernel) ble først utviklet av Linus Torvalds på 90-tallet da han studerte ved Universitetet i Helsinki. Linus ble interessert i utvikling av operativsystemer gjennom sine studier, men ble frustrert av datidens lisensbegrensninger på tilgjengelige operativsystemer. I 1991 begynte han å utvikle kjernen (kernel) til det som skulle bli til Linux-kjernen som vi kjenner den i dag. En operativsystemkjerne fungerer som et bindeledd mellom maskinvare og programmer. Den er ansvarlig for å starte systemet og å tildele ressurser til programmer. En kjerne er i seg selv ikke særlig nyttig uten resten av operativsystemet, slik som enhetsdrivere, brukerprogrammer, kompilatorer, osv. Heldigvis eksisterte allerede GNU-prosjektet, som også hadde som mål å tilby fri (som i frihet)\footnote{"Free as in freedom, not as in beer" - Richard Stallman, Grunnlegger av GNU Prosjektet} programvare, og i grunn bare manglet en velfungerende kjerne. Ved å kombinere Linux-kjernen med GNU-programvare ble Linux til et fullverdig operativsystem som mange i dag omtaler som "GNU/Linux". 

GNU/Linux kommer i dag som flere ulike varianter, eller distribusjoner, avhengig av hvilken programvare som brukes. Noen av de mest populære distribusjonene er Debian, Fedora Linux og Arch Linux. Datamaskinene dere skal bruke på lab bruker Ubuntu, som er en variant av Debian med et mer begynnervennlig brukergrensesnitt.

Et av kjerneprinsippene til både GNU- og Linux-prosjektet er at programvare, samt tilhørende kildekode, skal være åpent tilgjengelig og respektere brukerens friheter til å bruke, endre og (re)distriburere den. Dette gjelder også utviklingen av Linux. Hvem som helst kan bidra til Linux-kjernen, og kun et fåtall utviklere jobber med å vedlikeholde prosjektet profesjonelt. Projektenes dugnadsånd har vært en stor faktor i GNU/Linux sin utvikling, og holdningen har spredd seg til å bli en hjørnestein i kulturen til dagens programvareutviklere. Dette er også en av flere grunner til at programvareutviklere foretrekker å bruke Linux både til daglig bruk og profesjonelt. GNU/Linux brukes i dag av Android, ChromeOS, majoriteten av tjenere på internett og en stor andel av alle tilpassede datasystemer. SpaceX sin Falcon 9, NASA sin Perseverance og Tesla sine biler er bare noen få av alle nevneverdige Linux-brukere. 

% \end{alphasection}
% \setcounter{section}{0}

